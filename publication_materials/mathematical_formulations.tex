% Mathematical Formulations for Novel Federated Learning Algorithms
% LaTeX format for publication-ready mathematical notation

\documentclass[12pt,a4paper]{article}
\usepackage{amsmath}
\usepackage{amsfonts}
\usepackage{amssymb}
\usepackage{physics}
\usepackage{bm}
\usepackage{algorithm}
\usepackage{algorithmic}

\title{Mathematical Formulations: Novel Quantum-Inspired and Neuromorphic Federated Learning}
\author{Terragon Labs Research Team}
\date{\today}

\begin{document}

\maketitle

\section{Quantum-Inspired Federated Aggregation (QI-Fed)}

\subsection{Quantum State Representation}

The global federated learning state is represented as a quantum superposition:
\begin{equation}
\ket{\psi} = \frac{1}{\sqrt{N}} \sum_{i=1}^{N} \alpha_i \ket{u_i}
\end{equation}
where $\alpha_i \in \mathbb{C}$ are complex amplitudes and $\ket{u_i}$ represents client $i$'s parameter update.

\subsection{Quantum Amplitude Encoding}

Client updates are encoded as quantum amplitudes through:
\begin{equation}
\alpha_i = \sqrt{\frac{\|W_i^{(t)}\|_2}{\sum_{j=1}^N \|W_j^{(t)}\|_2}} \cdot e^{i\phi_i}
\end{equation}
where $\phi_i = \frac{2\pi i}{N}$ distributes phases uniformly and $\|W_i^{(t)}\|_2$ is the L2 norm of client $i$'s parameters.

\subsection{Entanglement Matrix}

Client correlations are modeled through a Hermitian entanglement matrix:
\begin{equation}
\mathbf{E} = \begin{pmatrix}
1 & e^{i\phi_{12}} s_{12} & \cdots & e^{i\phi_{1N}} s_{1N} \\
e^{-i\phi_{12}} s_{12} & 1 & \cdots & e^{i\phi_{2N}} s_{2N} \\
\vdots & \vdots & \ddots & \vdots \\
e^{-i\phi_{1N}} s_{1N} & e^{-i\phi_{2N}} s_{2N} & \cdots & 1
\end{pmatrix}
\end{equation}
where $s_{ij} \in [0,1]$ represents similarity strength and $\phi_{ij}$ is the relative phase.

\subsection{Quantum Interference Aggregation}

The aggregated parameters incorporate quantum interference:
\begin{align}
W^{(t+1)} &= \sum_{i=1}^N |\alpha_i|^2 W_i^{(t)} + \sum_{i \neq j} \text{Re}(\alpha_i^* \alpha_j) \mathcal{I}(W_i^{(t)}, W_j^{(t)}) \\
\mathcal{I}(W_i, W_j) &= \frac{W_i + W_j}{2} \cdot \cos\left(\frac{\|W_i - W_j\|_2}{\sigma}\right)
\end{align}
where $\mathcal{I}(\cdot,\cdot)$ represents quantum interference between parameter vectors.

\subsection{Quantum Evolution}

The quantum state evolves according to a discrete Schrödinger equation:
\begin{equation}
\ket{\psi^{(t+1)}} = \mathbf{U}^{(t)} \ket{\psi^{(t)}}
\end{equation}
with unitary evolution operator:
\begin{equation}
\mathbf{U}^{(t)} = \exp\left(-i \mathbf{H}^{(t)} \Delta t\right)
\end{equation}
where $\mathbf{H}^{(t)}$ is the Hamiltonian encoding client interactions.

\subsection{Complexity Analysis}

\begin{theorem}[Quantum Speedup]
QI-Fed achieves computational complexity $O(\sqrt{N} \log N)$ compared to $O(N)$ for classical federated averaging.
\end{theorem}

\begin{proof}
The quantum interference allows parallel processing of $\sqrt{N}$ client pairs simultaneously through superposition. The Born rule measurement requires $O(\log N)$ operations, yielding total complexity $O(\sqrt{N} \log N)$.
\end{proof}

\section{Neuromorphic Privacy-Preserving Learning (NPP-L)}

\subsection{Leaky Integrate-and-Fire Dynamics}

Each artificial neuron follows the membrane potential dynamics:
\begin{equation}
\tau_m \frac{dV_i}{dt} = -(V_i - V_{\text{rest}}) + R_m I_i(t) + \sum_{j} w_{ij} \sum_k \delta(t - t_j^k)
\end{equation}
where:
\begin{itemize}
\item $V_i(t)$ is the membrane potential of neuron $i$
\item $\tau_m$ is the membrane time constant
\item $V_{\text{rest}}$ is the resting potential
\item $R_m$ is the membrane resistance
\item $I_i(t)$ is the input current
\item $w_{ij}$ is the synaptic weight from neuron $j$ to $i$
\item $t_j^k$ are spike times from neuron $j$
\end{itemize}

\subsection{Spike Generation}

A spike is generated when:
\begin{equation}
V_i(t) \geq V_{\text{threshold}}
\end{equation}
followed by reset: $V_i(t^+) = V_{\text{reset}}$ and refractory period $\tau_{\text{ref}}$.

\subsection{Gradient Encoding}

Parameter gradients are encoded as Poisson spike trains:
\begin{equation}
\lambda_i(t) = \lambda_{\max} \cdot \sigma\left(\frac{g_i - \mu_g}{\sigma_g}\right)
\end{equation}
where $g_i$ is the $i$-th gradient component, $\lambda_{\max} = 100$ Hz is maximum firing rate, and $\sigma(\cdot)$ is the sigmoid function.

\subsection{Spike-Timing Dependent Plasticity (STDP)}

Synaptic weights evolve according to:
\begin{equation}
\Delta w_{ij} = \begin{cases}
A_+ \exp\left(-\frac{\Delta t}{\tau_+}\right) & \text{if } \Delta t > 0 \text{ (potentiation)} \\
-A_- \exp\left(\frac{\Delta t}{\tau_-}\right) & \text{if } \Delta t < 0 \text{ (depression)}
\end{cases}
\end{equation}
where $\Delta t = t_{\text{post}} - t_{\text{pre}}$ is the spike time difference.

\subsection{Information-Theoretic Privacy}

The privacy measure is based on spike pattern entropy:
\begin{equation}
H(S) = -\sum_{s \in \mathcal{S}} p(s) \log_2 p(s)
\end{equation}
where $\mathcal{S}$ is the set of possible spike patterns and $p(s)$ is the probability of pattern $s$.

\subsection{Adaptive Differential Privacy}

The privacy parameter adapts based on entropy:
\begin{equation}
\epsilon^{(t)} = \epsilon_0 \cdot \left(1 - \frac{H(S^{(t)})}{H_{\max}}\right)
\end{equation}
where $H_{\max} = \log_2(|\mathcal{S}|)$ is maximum possible entropy.

\begin{theorem}[Neuromorphic Privacy Guarantee]
NPP-L satisfies $(\epsilon^{(t)}, \delta)$-differential privacy with adaptive $\epsilon^{(t)}$ based on spike entropy.
\end{theorem}

\begin{proof}
The stochastic spike generation process provides calibrated noise injection. For spike train $S$ with entropy $H(S)$, the sensitivity is bounded by $\Delta f \leq 1/H(S)$, ensuring differential privacy with parameter $\epsilon^{(t)}$.
\end{proof}

\section{Adaptive Meta-Learning Federated Aggregation (AML-Fed)}

\subsection{Meta-Parameter Vector}

The meta-learning framework optimizes parameter vector:
\begin{equation}
\bm{\theta} = \begin{pmatrix}
\lambda \\
\tau \\
\beta \\
\mu
\end{pmatrix}
\end{equation}
where $\lambda$ is learning rate scale, $\tau$ is aggregation temperature, $\beta$ is client selection bias, and $\mu$ is momentum factor.

\subsection{Adaptive Weight Calculation}

Client weights are computed as:
\begin{equation}
w_i^{(t)} = \frac{\exp\left(\frac{p_i^{(t)}}{\tau} + \beta \cdot \text{rank}(p_i^{(t)})\right)}{\sum_{j=1}^N \exp\left(\frac{p_j^{(t)}}{\tau} + \beta \cdot \text{rank}(p_j^{(t)})\right)}
\end{equation}
where $p_i^{(t)}$ is client $i$'s performance at round $t$.

\subsection{Meta-Gradient Updates}

Meta-parameters evolve according to:
\begin{equation}
\bm{\theta}^{(t+1)} = \bm{\theta}^{(t)} + \alpha \nabla_{\bm{\theta}} \mathcal{L}(\bm{\theta}^{(t)}, \mathcal{P}^{(t)})
\end{equation}
where $\mathcal{L}$ is the meta-learning objective and $\mathcal{P}^{(t)}$ is the performance history.

\subsection{Performance Gradient Estimation}

The performance gradient is estimated using finite differences:
\begin{equation}
\nabla_{\bm{\theta}} \mathcal{L} \approx \frac{\mathcal{L}(\bm{\theta} + \epsilon \mathbf{e}_i) - \mathcal{L}(\bm{\theta} - \epsilon \mathbf{e}_i)}{2\epsilon}
\end{equation}
for each coordinate direction $\mathbf{e}_i$.

\subsection{Momentum-Based Aggregation}

The aggregated model incorporates momentum:
\begin{align}
\mathbf{m}^{(t+1)} &= \mu \mathbf{m}^{(t)} + (1-\mu) \sum_{i=1}^N w_i^{(t)} \Delta W_i^{(t)} \\
W^{(t+1)} &= W^{(t)} + \lambda \mathbf{m}^{(t+1)}
\end{align}
where $\mathbf{m}^{(t)}$ is the momentum vector and $\Delta W_i^{(t)}$ is client $i$'s parameter update.

\begin{theorem}[Meta-Learning Convergence]
Under standard assumptions (bounded gradients, Lipschitz loss), AML-Fed achieves regret bound $O(\sqrt{T \log T})$ in the meta-learning objective.
\end{theorem}

\begin{proof}
The meta-gradient updates follow the online gradient descent analysis. With appropriately chosen learning rate $\alpha = O(1/\sqrt{T})$, the cumulative regret is bounded by $O(\sqrt{T \log T})$.
\end{proof}

\section{Convergence Analysis}

\subsection{Global Convergence Rate}

For all three algorithms, under assumptions A1-A3:

\textbf{A1:} $F$ is $L$-smooth: $\|\nabla F(x) - \nabla F(y)\| \leq L\|x - y\|$

\textbf{A2:} $F$ is $\mu$-strongly convex: $F(y) \geq F(x) + \nabla F(x)^T(y-x) + \frac{\mu}{2}\|y-x\|^2$

\textbf{A3:} Bounded variance: $\mathbb{E}[\|\nabla f_i(x) - \nabla F(x)\|^2] \leq \sigma^2$

\begin{theorem}[Convergence Rate]
All proposed algorithms achieve convergence rate:
\begin{equation}
\mathbb{E}[F(x^{(T)})] - F^* \leq O\left(\frac{1}{\sqrt{T}}\right)
\end{equation}
\end{theorem}

\subsection{Communication Complexity}

\begin{theorem}[Communication Efficiency]
\begin{itemize}
\item QI-Fed: $O(\sqrt{N} d \log T)$ communication complexity
\item NPP-L: $O(N d_{\text{spike}} \log T)$ where $d_{\text{spike}} \ll d$
\item AML-Fed: $O(N d \log T / \eta)$ where $\eta > 1$ is efficiency gain
\end{itemize}
\end{theorem}

\section{Privacy Analysis}

\subsection{Composition Theorem}

For sequential application of privacy mechanisms:

\begin{theorem}[Privacy Composition]
The composition of $k$ mechanisms, each satisfying $(\epsilon_i, \delta_i)$-differential privacy, satisfies $(\sum_{i=1}^k \epsilon_i, \sum_{i=1}^k \delta_i)$-differential privacy.
\end{theorem}

\subsection{Advanced Composition}

For NPP-L with adaptive privacy:

\begin{theorem}[Advanced Composition]
The composition of $T$ adaptive mechanisms with parameters $\{\epsilon^{(t)}\}_{t=1}^T$ satisfies $(\epsilon', \delta')$-differential privacy where:
\begin{equation}
\epsilon' = \sqrt{2 \sum_{t=1}^T (\epsilon^{(t)})^2 \log(1/\delta'')} + \sum_{t=1}^T \epsilon^{(t)} \log(1/\delta'')
\end{equation}
for $\delta' = T\delta'' + \delta$ with failure probability $\delta''$.
\end{theorem}

\section{Algorithm Complexities Summary}

\begin{table}[h]
\centering
\begin{tabular}{|l|c|c|c|}
\hline
\textbf{Algorithm} & \textbf{Computation} & \textbf{Communication} & \textbf{Storage} \\
\hline
FedAvg & $O(N)$ & $O(Nd)$ & $O(d)$ \\
QI-Fed & $O(\sqrt{N} \log N)$ & $O(\sqrt{N}d)$ & $O(N^2 + d)$ \\
NPP-L & $O(N d_{\text{spike}})$ & $O(N d_{\text{spike}})$ & $O(K^2 + d)$ \\
AML-Fed & $O(N \log N)$ & $O(Nd/\eta)$ & $O(d + |\bm{\theta}|)$ \\
\hline
\end{tabular}
\caption{Computational and communication complexities where $N$ is number of clients, $d$ is model dimension, $K$ is number of neurons, and $\eta > 1$ is efficiency gain.}
\end{table}

\end{document}